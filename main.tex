%%%%%%%%%%%%%%%%%
% This is an example CV created using altacv.cls (v1.6.4, 13 Nov 2021) written by
% LianTze Lim (liantze@gmail.com), based on the
% Cv created by BusinessInsider at http://www.businessinsider.my/a-sample-resume-for-marissa-mayer-2016-7/?r=US&IR=T
%
%% It may be distributed and/or modified under the
%% conditions of the LaTeX Project Public License, either version 1.3
%% of this license or (at your option) any later version.
%% The latest version of this license is in
%%    http://www.latex-project.org/lppl.txt
%% and version 1.3 or later is part of all distributions of LaTeX
%% version 2003/12/01 or later.
%%%%%%%%%%%%%%%%

%% Use the "normalphoto" option if you want a normal photo instead of cropped to a circle
% \documentclass[10pt,a4paper,normalphoto]{altacv}

\documentclass[10pt,a4paper,ragged2e,withhyper]{altacv}

%% AltaCV uses the fontawesome5 package.
%% See http://texdoc.net/pkg/fontawesome5 for full list of symbols.

% Change the page layout if you need to
\geometry{left=1.25cm,right=1.25cm,top=1.5cm,bottom=1.5cm,columnsep=1.2cm}

% The paracol package lets you typeset columns of text in parallel
\usepackage{paracol}
% \usepackage{biblatex}


% \usepackage[english,polish]{babel}


% Change the font if you want to, depending on whether
% you're using pdflatex or xelatex/lualatex
\ifxetexorluatex
  % If using xelatex or lualatex:
  \setmainfont{Lato}
\else
  % If using pdflatex:
  \usepackage[default]{lato}
\fi

% Change the colours if you want to
\definecolor{VividPurple}{HTML}{1e6091}
\definecolor{SlateGrey}{HTML}{2E2E2E}
\definecolor{LightGrey}{HTML}{666666}
\definecolor{Accent}{HTML}{184e77}
% \colorlet{name}{black}
% \colorlet{tagline}{PastelRed}
\colorlet{heading}{VividPurple}
\colorlet{headingrule}{VividPurple}
% \colorlet{subheading}{PastelRed}
\colorlet{accent}{VividPurple}
\colorlet{emphasis}{SlateGrey}
\colorlet{body}{LightGrey}

% Change some fonts, if necessary
% \renewcommand{\namefont}{\Huge\rmfamily\bfseries}
% \renewcommand{\personalinfofont}{\footnotesize}
% \renewcommand{\cvsectionfont}{\LARGE\rmfamily\bfseries}
% \renewcommand{\cvsubsectionfont}{\large\bfseries}

% Change the bullets for itemize and rating marker
% for \cvskill if you want to
\renewcommand{\itemmarker}{{\small\textbullet}}
\renewcommand{\ratingmarker}{\faCircle}

%% Use (and optionally edit if necessary) this .tex if you
%% want to use an author-year reference style like APA(6)
%% for your publication list
% When using APA6 if you need more author names to be listed
% because you're e.g. the 12th author, add apamaxprtauth=12
% \usepackage[backend=biber,style=apa6,sorting=ydnt]{biblatex}
\usepackage[backend=biber,style=mla,sorting=ydnt,minbibnames=2]{biblatex}
\defbibheading{pubtype}{\cvsubsection{#1}}
\renewcommand{\bibsetup}{\vspace*{-\baselineskip}}
\AtEveryBibitem{\makebox[\bibhang][l]{\itemmarker}}
\setlength{\bibitemsep}{0.25\baselineskip}
\setlength{\bibhang}{1.25em}

% Make author name bold: https://tex.stackexchange.com/a/327046
\newcommand*{\boldname}[3]{%
  \def\lastname{#1}%
  \def\firstname{#2}%
  \def\firstinit{#3}}
\boldname{}{}{}

\renewcommand{\mkbibnamegiven}[1]{%
  \ifboolexpr{ ( test {\ifdefequal{\firstname}{\namepartgiven}} or test {\ifdefequal{\firstinit}{\namepartgiven}} ) and test {\ifdefequal{\lastname}{\namepartfamily}} }
  {\mkbibbold{#1}}{#1}%
}

\renewcommand{\mkbibnamefamily}[1]{%
  \ifboolexpr{ ( test {\ifdefequal{\firstname}{\namepartgiven}} or test {\ifdefequal{\firstinit}{\namepartgiven}} ) and test {\ifdefequal{\lastname}{\namepartfamily}} }
  {\mkbibbold{#1}}{#1}%
}
\boldname{Stępień}{Piotr}{}

%% Use (and optionally edit if necessary) this .tex if you
%% want an originally numerical reference style like IEEE
%% for your publication list
% \input{pubs-num}

%% sample.bib contains your publications
\addbibresource{sample.bib}

\begin{document}
\name{Piotr Stępień}
\tagline{Optomechatronic engineer, PhD in spe}
% Cropped to square from https://en.wikipedia.org/wiki/Marissa_Mayer#/media/File:Marissa_Mayer_May_2014_(cropped).jpg, CC-BY 2.0
%% You can add multiple photos on the left or right
\photoR{3cm}{11613_square}
% \photoL{2cm}{Yacht_High,Suitcase_High}
\personalinfo{%
  % Not all of these are required!
  % You can add your own with \printinfo{symbol}{detail}
  \email{piotr1stepien@gmail.com}
  \phone{887 623 317}
%   \mailaddress{ul. Mińska 71/263}
%   \location{03-828, Warsaw}
%   \homepage{marissamayr.tumblr.com}
%   \twitter{@marissamayer}
  \linkedin{piotr1stepien}
  \github{PioSte}
  \orcid{0000-0002-7604-4953}
  %% You can add your own arbitrary detail with
  %% \printinfo{symbol}{detail}[optional hyperlink prefix]
  % \printinfo{\faPaw}{Hey ho!}
  %% Or you can declare your own field with
  %% \NewInfoFiled{fieldname}{symbol}[optional hyperlink prefix] and use it:
  % \NewInfoField{gitlab}{\faGitlab}[https://gitlab.com/]
  % \gitlab{your_id}
	%%
  %% For services and platforms like Mastodon where there isn't a
  %% straightforward relation between the user ID/nickname and the hyperlink,
  %% you can use \printinfo directly e.g.
  % \printinfo{\faMastodon}{@username@instace}[https://instance.url/@username]
  %% But if you absolutely want to create new dedicated info fields for
  %% such platforms, then use \NewInfoField* with a star:
  % \NewInfoField*{mastodon}{\faMastodon}
  %% then you can use \mastodon, with TWO arguments where the 2nd argument is
  %% the full hyperlink.
  % \mastodon{@username@instance}{https://instance.url/@username}
}

\makecvheader

%% Depending on your tastes, you may want to make fonts of itemize environments slightly smaller
\AtBeginEnvironment{itemize}{\small}

%% Set the left/right column width ratio to 6:4.
\columnratio{0.6}

% Start a 2-column paracol. Both the left and right columns will automatically
% break across pages if things get too long.
\begin{paracol}{2}

\cvsection{Experience}

\cvevent{Optomechatronic engineer}{Warsaw University of Technology}{August 2021 -- Ongoing}{Warsaw, PL}
\begin{itemize}
\item Horizon 2020 project "REVEAL: Neuronal microscopy for cell behavioural examination and manipulation".
\item Holographic microscope design and full hardware and software integration.
\item Technologies: Python, NumPy, scikit-image, Cupy, Arduino, Autodesk Inventor.
\end{itemize}

\divider

\cvevent{Scholarship, PhD student}{Warsaw University of Technology}{May 2017 -- August 2021}{Warsaw, PL}
\begin{itemize}
\item FNP Team-TECH project "BiOpTo: Tomographic phase microscope for biomedical applications".
\item Conception, design and hardware/software integration of quantitative phase image stitching in the holographic microscope setup.
\item Design of hologram compression processing path.
\item Technologies: Python, NumPy, scikit-image, Pyculib, Matlab, Autodesk Inventor, Latex.
\item Skills: numerical Fourier analysis, image processing, tomographic reconstruction.
\end{itemize}

\divider

\cvevent{Optomechatronic engineer}{Astri Polska}{May 2016 --  April 2017}{Warsaw, PL}
\begin{itemize}
\item Ray tracing software development, numerical phase aberration correction, optical hardware selection, measurement uncertainty estimation, mechanical design.
\item Technologies: Matlab/Octave, Autodesk Inventor.
\end{itemize}

\divider

\cvevent{M.S. student scholarship}{Warsaw University of Technology}{April 2015 -- May 2015}{Warsaw, PL}

\begin{itemize}
\item FNP Team project "3DPhase -- Phase Microscopy and Tomography"
\item Software integration of the holographic tomographs hardware components, assembling and alignment of the holographic tomograph, CAD modeling of the holographic tomograph.
\item Technologies: Matlab, Autodesk Inventor.
\end{itemize}

% \divider

% \cvevent{Product Engineer}{Google}{23 June 1999 -- 2001}{Palo Alto, CA}

% \begin{itemize}
% \item Joined the company as employe \#20 and female employee \#1
% \item Developed targeted advertisement in order to use user's search queries and show them related ads
% \end{itemize}

% \cvsection{A Day of My Life}

% % Adapted from @Jake's answer from http://tex.stackexchange.com/a/82729/226
% % \wheelchart{outer radius}{inner radius}{
% % comma-separated list of value/text width/color/detail}
% % Some ad-hoc tweaking to adjust the labels so that they don't overlap
% \hspace*{-1em}  %% quick hack to move the wheelchart a bit left
% \wheelchart{1.5cm}{0.5cm}{%
%   10/13em/accent!30/Sleeping \& dreaming about work,
%   25/9em/accent!60/Public resolving issues with Yahoo!\ investors,
%   5/11em/accent!10/\footnotesize\\[1ex]New York \& San Francisco Ballet Jawbone board member,
%   20/11em/accent!40/Spending time with family,
%   5/8em/accent!20/\footnotesize Business development for Yahoo!\ after the Verizon acquisition,
%   30/9em/accent/Showing Yahoo!\ \mbox{employees} that their work has meaning,
%   5/8em/accent!20/Baking cupcakes
% }

% use ONLY \newpage if you want to force a page break for
% ONLY the currentc column
\newpage

\cvsection{Publications}

\nocite{*}

\printbibliography[heading=pubtype,title={\printinfo{\faBook}{Books}},type=book]

\divider

\printbibliography[heading=pubtype,title={\printinfo{\faFile*[regular]}{Journal Articles}}, type=article]

\divider

\printbibliography[heading=pubtype,title={\printinfo{\faUsers}{Conference Proceedings}},type=inproceedings]

%% Switch to the right column. This will now automatically move to the second
%% page if the content is too long.
\switchcolumn

% \cvsection{Life Philosophy}
% \begin{quote}
% ``If you don't have any shadows, you're not standing in the light.''
% \end{quote}

\cvsection{Education}

\cvevent{PhD\ in Photonics Engineering}{Warsaw University of Technology, Faculty of Mechatronics}{2017 -- 2022}{}
"Quantitative phase imaging of biological structures in digital holographic microscopy and tomography with extended field of view."
% Ilościowe obrazowanie fazowe struktur biologicznych w optycznej mikroskopii i tomografii holograficznej z rozszerzonym polem widzenia

\divider

\cvevent{MSc\ in Photonics Engineering}{Warsaw University of Technology, Faculty of Mechatronics}{2015 -- 2016}{}
% Budowa systemu mikroskopu fazowego z oświetleniem LED i przestrzennym modulatorem intensywności.
"Design of phase microscope with frequency modulation based on spatial light modulator and with LED illumination."

\divider

\cvevent{Eng.\ in Photonics Engineering}{Warsaw University of Technology, Faculty of Mechatronics}{2011 -- 2015}{}
"The construction of the self-interference holographic tomography setup."


\cvsection{Most Proud of}

% \cvachievement{\faTrophy}{Courage I had}{to take a sinking ship and try to make it float.}

% \divider

\cvachievement{\faTrophy}{Best accuracy CNN classifier}{during the International Summer School on Deep Learning in Gdańsk (2018).}

\divider

\cvachievement{\faCloud}{Self-hosted cloud}{Nextcloud, Plex and PiHole set up with Docker.}

\divider

\cvachievement{\faHeartbeat}{Persistence}{I showed despite the hard moments during my PhD.}

% \divider

% \cvachievement{\faFemale}{Inspiring women in tech}{Youngest CEO on Fortune's list of 50 most powerful women.}

\cvsection{Strengths}

\cvtag{Inquisitive}
\cvtag{Creative}
\cvtag{Quick learner}
\cvtag{Versatile}
\cvtag{Attention to detail}

% \divider\smallskip

% \cvtag{UX}
% \cvtag{Mobile Devices \& Applications}
% \cvtag{Product Management \& Marketing}

\cvsection{Languages}

\cvskill{English}{4.5}
% \divider

% \cvskill{Spanish}{4}
% \divider

\cvskill{German}{1} %% supports X.5 values.

\newpage

\cvsection{Referees}

% \cvref{name}{email}{mailing address}
\cvref{Prof.\ dr hab. inż. Małgorzata Kujawińska}{Institute of Micromechanics\\\hspace{2mm}and Photonics}{malgorzata.kujawinska@pw.edu.pl}
\hspace{6.7mm}{ul. Św. Andrzeja Boboli 8\\\hspace{6.7mm}Warsaw 02-525\\\hspace{6.7mm}room 515}
% {ul. Św. Andrzeja Boboli 8\\Warsaw 02-525\\room 515}

\divider

\cvref{Dr\ inż. Wojciech Krauze}{Institute of Micromechanics\\\hspace{2mm}and Photonics}{wojciech.krauze@pw.edu.pl}
\hspace{6.7mm}{ul. Św. Andrzeja Boboli 8\\\hspace{6.7mm}Warsaw 02-525\\\hspace{6.7mm}room 520A}
% {ul. Św. Andrzeja Boboli 8\\Warsaw 02-525\\room 515}

\end{paracol}

\end{document}
